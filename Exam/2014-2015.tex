\documentclass[12pt,a4paper]{article}
\usepackage{amsmath,amssymb}
\usepackage{graphicx}
\usepackage{float}
\usepackage{tikz}
\usepackage[english]{babel}
\usepackage[utf8]{inputenc}
\usetikzlibrary{shapes, arrows, positioning}


\usepackage{amsmath,amssymb,mdframed}            % AMS package gives better equation layouts
\setcounter{page}{2}                    % sets first page number to 2
\setlength{\oddsidemargin}{-0.25in}     % set left margin
\setlength{\textwidth}{6.5in}           % set text width
\setlength{\topmargin}{-0.5in}          % controls layout at
\setlength{\headsep}{1cm}             % top of page
\setlength{\textheight}{9.0in}          % set text length



\makeatletter
\renewcommand{\@oddhead}{\hfill MA3600/15}  % sets header
\renewcommand{\@oddfoot}{\hfil \arabic{page} \hfil}    % sets page footer
\makeatother

\renewcommand{\labelenumi}{\arabic{enumi}} % Sets the first level of enumerate to be arabic (normal) numbers
\renewcommand{\labelenumii}{(\alph{enumii})} %Sets the second level of enumerate to be (a), (b), (c), .....
\renewcommand{\labelenumiii}{(\roman{enumiii})} % Sets the third level of enumerate to be (i), (ii), (iii), ....



\begin{document}
\begin{enumerate}
\setcounter{enumi}{3}

\renewcommand\labelenumi{\bfseries\theenumi.}

\item

    \begin{enumerate}
        \item Provide definitions for the following terms:
            \begin{itemize}
                \item Normal form game.

                \item Strictly dominated strategy.

                \item Weakly dominated strategy.

                \item Best response strategy.

                \item Nash equilibrium.

                    ~\hfill{[5]}
            \end{itemize}

        \item     Consider the following game:

            \[\begin{pmatrix}
            (7,3) & (0,2)\\
            (2,0) & (6,2)\\
            \end{pmatrix}\]

            \begin{enumerate}
            \item By clearly stating the techniques used, obtain all (if any) pure Nash equilibria.

            ~\hfill{[4]}

            \item Sketch the utilities to player 1 (the row player) assuming that the 2nd player (the column player) plays a mixed strategy: $\sigma_2 = (y,1-y)$.

            ~\hfill{[2]}

            \item Sketch the utilities to player 2 (the column player) assuming that the 1st player (the row player) plays a mixed strategy: $\sigma_1 = (x,1-x)$.

            ~\hfill{[2]}

            \item State, prove and use the Equality of Payoffs theorem to obtain all Nash equilibria for the game.

            ~\hfill{[6]}

            \item Consider the same game with an extra strategy for the row player:

                \[\begin{pmatrix}
                (7,3) & (0,2)\\
                (3,1) & (3,1)\\
                (2,0) & (6,2)\\
                \end{pmatrix}\]

                By directly calculating the set of best response strategies \(B_1\) for the row player, obtain all Nash equilibria for this new game.
                State any theorem(s) used.

            ~\hfill{[6]}

        \end{enumerate}
    \end{enumerate}

\newpage
\item
    This question considers evolutionary population games.
    Throughout, the following game is considered:

    Road users in a given country can choose to drive on either the left (\(L\)) side or the right (\(R\)) side of the road.
    The strategy set in this game is \(S=\{L, R\}\).

    If all users drive on the same side of the road then no accidents will occur.
    If users drive on the opposite side and meet each other then they may have an accident.

    Considering a population vertor \(\chi=(x,1-x)\) where \(x\) is the proportion of the population using strategy \(L\), the utilities are given by:

    \[u(L,\chi)=1+x\]
    and
    \[u(R,\chi)=1+(1-x)\]


    \begin{enumerate}

            \item Define a stable strategy in a population game.

            \hfill[2]

            \item State and prove a theorem giving a necessary condition for stable strategies.
                Use this theorem to obtain all potential evolutionary stable strategies in the described game.

            \hfill[6]

            \item Define a post entry population.

            \hfill[2]

            \item Define an evolutionary stable strategy.

            \hfill[2]


            \item Obtain all evolutionary stable strategies for the described game.

            \hfill[12]

            \item Offer an interpretation for the answer to question (e).

            \hfill[1]
    \end{enumerate}

\newpage
\item

    \begin{enumerate}
        \item Define a characteristic function game \(G=(N,v)\).

        \hfill{[2]}

        \item Define the Shapley value.

        \hfill{[2]}

        \item Obtain the Shapley value for the following characteristic function games:

            \[
                v_1(c) = \begin{cases}
                    6,& \text{if }c=\{1\}\\
                    6,& \text{if } c=\{2\}\\
                    7,& \text{if } c=\{3\}\\
                    7,& \text{if } c=\{1,2\}\\
                    7,& \text{if } c=\{2,3\}\\
                    20,& \text{if } c=\{1,3\}\\
                    40,& \text{if } c=\{1,2,3\}\\
                \end{cases}
            \]

            \[
                v_2(c) = \begin{cases}
                    100,& \text{if }c=\{1\}\\
                    6,& \text{if } c=\{2\}\\
                    7,& \text{if } c=\{3\}\\
                    100,& \text{if } c=\{1,2\}\\
                    7,& \text{if } c=\{2,3\}\\
                    100,& \text{if } c=\{1,3\}\\
                    100,& \text{if } c=\{1,2,3\}\\
                \end{cases}
            \]

        \hfill{[8]}

        \item  For a given characteristic function game \(G=(N,v)\) a payoff vector \(\lambda\) is efficient if:

            \[\sum_{i=1}^{N}\lambda_i=v(\Omega)\]

            Prove that the Shapley value is efficient.

        \hfill{[6]}

        \item For \(G(N,v)\) a payoff vector \(\lambda\) is symmetric if the following holds:

        If \(v(C\cup i)=v(C\cup j)\) for all \(C\in 2^{\Omega}\setminus\{i,j\}\) then \(x_i=x_j\).

        Prove that the Shapley value is symmetric.

        \hfill{[7]}

    \end{enumerate}


\end{enumerate}

\makeatletter
\renewcommand{\@oddfoot}{\hfil \arabic{page}X \hfil}    % sets last page footer
\makeatother

\end{document}
