\documentclass[xcolor=table]{beamer}

\usepackage{amsmath}
\usepackage{booktabs}
\usepackage{hyperref}
\usepackage[table]{xcolor}
\usepackage{tikz}
\usepackage{graphics}
\usetikzlibrary{calc, shapes}

\setbeamertemplate{navigation symbols}{}%remove navigation symbols

\title{Connection between Nash flows an optimal flows}
\subtitle{Game Theory}
\author{Vincent Knight}
\date{}

\begin{document}

\frame{\titlepage}

\frame{
$$(G,r,c)$$
\begin{itemize}
    \item $G=(V,E)$, with a defined set of sources $s_i$ and sinks $t_i$;
    \item A commodity $r_i$;
    \item A set of latencies: $c_e$.
\end{itemize}
}

\frame{

\tikzstyle{vertex} = [draw, shape=circle, minimum width=.3cm, inner sep=.5pt]
\begin{center}
\begin{tikzpicture}[sloped]
    \draw (0,.5) node[vertex] (s1) {\tiny{$s_1$}};
    \draw (5,.5) node[vertex] (t) {\tiny{$t$}};
    \draw (s1) edge[out=-45,in=-135,->] node [below] {\tiny{$c(x)=x$}} node [above,red] {\tiny$1-\alpha$} (t);
    \draw (s1) edge[out=45,in=135,->] node [above] {\tiny{$c(x)=1$}} node [below,red] {\tiny$\alpha$} (t);
    \node at (s1) [left=.2cm] {$\tiny{1}$};
\end{tikzpicture}
\end{center}

\pause

Define the potential function:

$$\Phi(f)=\sum_{e\in E}\int_{0}^{f_e}c_e(x)dx$$

\pause

$$\Phi(f)=\alpha+\frac{(1-\alpha)^2}{2}=\frac{1}{2}-\frac{\alpha^2}{2}$$

$$f^*=(0,1)\text{ minimises }\Phi(f)$$

}

\frame{

\tikzstyle{vertex} = [draw, shape=circle, minimum width=.3cm, inner sep=.5pt]
\begin{center}
\begin{tikzpicture}[sloped]
    \draw (0,.5) node[vertex] (s1) {\tiny{$s_1$}};
    \draw (5,.5) node[vertex] (t) {\tiny{$t$}};
    \draw (s1) edge[out=-45,in=-135,->] node [below] {\tiny{$c(x)=x$}} node [above,red] {\tiny$1-\alpha$} (t);
    \draw (s1) edge[out=45,in=135,->] node [above] {\tiny{$c(x)=1$}} node [below,red] {\tiny$\alpha$} (t);
    \node at (s1) [left=.2cm] {$\tiny{1}$};
\end{tikzpicture}
\end{center}

\pause

Define marginal costs:

$$c^*=\frac{d}{dx}(xc(x))$$

\pause

\begin{center}
\begin{tikzpicture}[sloped]
    \draw (0,.5) node[vertex] (s1) {\tiny{$s_1$}};
    \draw (5,.5) node[vertex] (t) {\tiny{$t$}};
    \draw (s1) edge[out=-45,in=-135,->] node [below] {\tiny{$c(x)=2x$}} node [above,red] {\tiny$1-\alpha$} (t);
    \draw (s1) edge[out=45,in=135,->] node [above] {\tiny{$c(x)=1$}} node [below,red] {\tiny$\alpha$} (t);
    \node at (s1) [left=.2cm] {$\tiny{1}$};
\end{tikzpicture}
\end{center}

}

\frame{
\tikzstyle{vertex} = [draw, shape=circle, minimum width=.3cm, inner sep=.5pt]
\begin{center}
\begin{tikzpicture}[sloped]
    \draw (0,.5) node[vertex] (s1) {\tiny{$s_1$}};
    \draw (5,.5) node[vertex] (t) {\tiny{$t$}};
    \draw (s1) edge[out=-45,in=-135,->] node [below] {\tiny{$c(x)=2x$}} node [above,red] {\tiny$1-\alpha$} (t);
    \draw (s1) edge[out=45,in=135,->] node [above] {\tiny{$c(x)=1$}} node [below,red] {\tiny$\alpha$} (t);
    \node at (s1) [left=.2cm] {$\tiny{1}$};
\end{tikzpicture}
\end{center}

\pause

$$\tilde f=(.5,.5)\text{ is a Nash flow}$$

}

\frame{
\textbf{Theorem:} A feasible flow $\tilde f$ is a Nash flow for $(G,r,c)$ if and only if $\tilde f$ minimises $\Phi(f)$.

\vspace{1cm}

\textbf{Theorem:} A feasible flow $f^*$ is an optimal flow for $(G,r,c)$ if and only if $f^*$ is a Nash flow for $(G,r,c^*)$.
}

\end{document}
